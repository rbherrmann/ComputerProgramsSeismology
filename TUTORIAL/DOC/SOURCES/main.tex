\documentclass{article}
\usepackage{amsmath}% if you are using this package,
                      % it must be loaded before amsthm.sty
\usepackage{amsthm}
\usepackage{float}
\usepackage{graphicx}
\usepackage{caption}
% \usepackage{subcaption}

\begin{document}

\title{Green's functions in Computer Programs in Seismology
{}}
\author{Robert B. Herrmann}

\maketitle

\begin{abstract}

The abstract text goes here.
\end{abstract}

\section{Introduction}

Computer Programs in Seismology provides codes for the computation of Grene's functions for layered isotropic and transverse isotropic media.
Confusion oftne arises in the use of the program for incorporating the source time function.

\section{Green's functions}


The equations of motion describing the displacement $ {\bf u} $ at a point within an infinite medium under the action of
an external force per unit
volume, $( F_1 , F_{2,} F_3 ) $, 
are
\begin{equation} \label{eq5_2_1}
\begin{aligned}
 \rho \frac{ \partial^2 { u_1 }  }{  \partial t^2 } &= 
{ \frac{ \partial { p_{11} }  }{  \partial x_1 } } +
{ \frac{ \partial { p_{12} }  }{  \partial x_2 } }+
{ \frac{ \partial { p_{13} }  }{  \partial x_3 } }+
{ F_1 } \\
 \rho \frac{ \partial^2 { u_2 }  }{  \partial t^2 } &=
{ \frac{ \partial { p_{21} }  }{  \partial x_1 } } +
{ \frac{ \partial { p_{22} }  }{  \partial x_2 } }+
{ \frac{ \partial { p_{23} }  }{  \partial x_3 } }+
{ F_2 } \\
 \rho \frac{ \partial^2 { u_3 }  }{  \partial t^2 } &=
{ \frac{ \partial { p_{31} }  }{  \partial x_1 } } +
{ \frac{ \partial { p_{32} }  }{  \partial x_2 } }+
{ \frac{ \partial { p_{33} }  }{  \partial x_3 } } +
{ F_3 } \\
\end{aligned}
\end{equation}
where $ \rho $ is the density.
The physical dimensions  of the quantities are $ u_i $ $( L)$,
$ p_{ij} $ $( M L^{-1} T^{-2}  ) $, $ F_i $ $( M L^{-2} T^{-2}  ) $ and $ \rho ( M L^{-3} )$.

If the body force in the
elastic wave equation is given by
$$
\left( \mathscr{F}_1 (t) {\bf e}_1 +  \mathscr{F}_2 (t) {\bf e}_2 +  \mathscr{F}_3 (t) {\bf e}_3 \right) \, \delta ( x_1 ) \delta ( x_2 ) \delta ( x_3 )  ,
$$
the
displacements in the space-angular frequency domain
are just
\begin{equation} \label{eq5_3_18}
u_i \,=\,  G_{ij} ( \omega ) \mathscr{F}_j ( \omega )
\end{equation}
where the $ G_{ij} ( \omega ) $ $(M^{-1} T^2 )$ are the  $i'th$ component of displacement for a point force
in the $j'th$ direction and are  given by
$$ 
4 \pi \rho G_{ij} ( {\bf x,} \omega ) \ =\    \,\,\,\,
 ( 3 \gamma_i \gamma_j \,-\, \delta_{ij} ) \exp ( \,-\, i \omega  R \,/\, \alpha ) \,/\, (  \,-\, \omega^2 R^3 ) 
$$
$$ 
\,+\, ( 3 \gamma_i \gamma_j \,-\, \delta_{ij} ) \exp (  \,-\, i \omega R \,/\, \alpha ) \,/\, ( i \omega \alpha R^2 )
$$
$$ 
\,+\, \gamma_i \gamma_j \exp ( \,-\, i \omega R \,/\, \alpha )\,/\, ( \alpha^2 R )
$$
\begin{equation} \label{eq5_3_19}
   \,-\, ( 3 \gamma_i \gamma_j \,-\, \delta_{ij} ) \exp (  \,-\, i \omega R \,/\, \beta ) \,/\, ( \,-\,\omega^2 R^3  )
\end{equation}
$$
\,-\, ( 3 \gamma_i \gamma_j \,-\, \delta_{ij} ) \exp (  \,-\, i \omega  R \,/\, \beta ) \,/\, (  i \omega \beta R^2 )
$$
$$
\left . 
\,-\,   \gamma_i \gamma_j   \exp (  \,-\, i \omega R \,/\, \beta )\,/\, ( \beta^2 R )  \right .
$$
$$
\left. 
\,+\,   \delta_{ij}   \exp (  \,-\, i \omega R \,/\, \beta )\,/\, ( \beta^2 R )  \right .
$$

Combining the results we have the following expression for the
displacement due to an arbitrary combination of dipoles and couples:
$$
u_i 
\,\equiv\, {\mathscr{M}}_{jk}   G_{ij,k} 
\,=\,  {\mathscr{M}}_{jk} \, \frac{ \partial G_{ij} ( {\bf x} , {\bf y} , \omega )   }{  \partial y_k } 
\,=\,  \,-\, {\mathscr{M}}_{jk} \, \frac{ \partial G_{ij} ( {\bf x} , {\bf y} , \omega )   }{  \partial x_k } 
%\eqno (\*[chap].\*[sec].5)
$$
where the units of $ {\mathscr{M}}_{ij} ( \omega )  $ are $(M L^2 T^{-1} )$ and of $ G_{ki,j} $ are $(M^{-1} L^{-1} T^2 )$. 
The explicit expression for $ G_{ij,k}$ is taken from Haskell (1963):
$$
4 \pi \rho ( i \omega )^2 G_{ij,k}  \,=\, 
- ( 3 \gamma_i \delta_{jk} + 3 \gamma_j \delta_{ik} + 3 \gamma_k \delta_{ij} - 15 \gamma_i \gamma_j \gamma_k ) 
\exp ( - i \omega R / \alpha ) (1 / R^4 )
$$
$$
 \ \ -\  ( 3 \gamma_i \delta_{jk} + 3 \gamma_j \delta_{ik} + 3 \gamma_k \delta_{ij} - 15 \gamma_i \gamma_j \gamma_k )
\exp ( - i \omega R / \alpha ) (1 / R^3 ) ( i \omega / \alpha )
$$
$$
 \ \ -\   ( \gamma_i \delta_{jk} + \gamma_j \delta_{ik} + \gamma_k \delta_{ij} - 6 \gamma_i \gamma_j \gamma_k  )
\exp ( - i \omega R / \alpha ) (1 / R^2 ) ( i \omega / \alpha )^2
$$
$$
 \ \ +\  \gamma_i \gamma_j \gamma_k 
\exp ( - i \omega R / \alpha ) (1 / R  ) ( i \omega / \alpha )^3
%\eqno (\*[chap].\*[sec].6)
$$
$$
 \ \ +\ 
 ( 3 \gamma_i \delta_{jk} + 3 \gamma_j \delta_{ik} + 3 \gamma_k \delta_{ij} - 15 \gamma_i \gamma_j \gamma_k ) 
\exp ( - i \omega R / \beta ) (1 / R^4 )
$$
$$
 \ \ +\   ( 3 \gamma_i \delta_{jk} + 3 \gamma_j \delta_{ik} + 3 \gamma_k \delta_{ij} - 15 \gamma_i \gamma_j \gamma_k )
\exp ( - i \omega R / \beta ) (1 / R^3 ) ( i \omega / \beta )
$$
$$
 \ \ +\   ( \gamma_i \delta_{jk} + \gamma_j \delta_{ik} + 2 \gamma_k \delta_{ij} - 6 \gamma_i \gamma_j \gamma_k  )
\exp ( - i \omega R / \beta ) (1 / R^2 ) ( i \omega / \beta )^2
$$
$$
 \ \ -\  ( \gamma_i \gamma_j \gamma_k - \gamma_k \delta_{ij} )
\exp ( - i \omega R / \beta ) (1 / R  ) ( i \omega / \beta )^3
$$
Figure \ref{F5_1} presents the elemental couple and dipole force
systems denoted by the 9 $ {\mathscr{M}}_{ij} $ terms. 
The $ {\mathscr{M}}_{ij} $ introduced here corresponds to the Fourier transform of moment.
\begin{figure}
\begin{center}
\includegraphics[scale=0.7]{Figures/nFIG4.eps}
\end{center}
\caption{ Basic second-order multi-pole force combinations.}
\label{F5_1}
\end{figure}

After converting from a cartesian to a cylindrical coordinate system using the transformation

The vector displacement components $ u_r $, $ u_{\phi} $ and $ u_z $
in a cylindrical coordinate system are related to those in a Cartesian system, with $u_3$ positive down,
$ u_1 $, $ u_2 $ and $ u_3 $ by the simple relations
\begin{align*}
u_r  \,&=\, \ \   u_1 \cos \phi \,+\, u_2 \sin \phi \\
u_{\phi}  \,&=\, -  u_1 \sin \phi  \,+\, u_2 \cos \phi \\
%\eqno (\*[chap].\*[sec].1)
u_z  \,&=\, u_3
\end{align*}
where
$ x_1 / r \,=\, \cos \phi $, $ x_2  / r \,=\, \sin \phi $,
$ r^2 \,=\, x_1^2 \,+\, x_2^2 $ and
$ R^2 \,=\, r^2 \,+\, z^2 $. In addition
$ \gamma_1 \,=\, x_1 / R \,=\, ( r / R ) \cos \phi$, 
$ \gamma_2 \,=\, x_2 / R \,=\, ( r / R ) \sin \phi $ and $ \gamma_3 \,=\, ( z / R ) $.

Using the summetry of $ M_{ij}$,
and combining terms in order of increasing azimuthal dependence,
we have the following expression for the displacement for point forces/moments in space with a Diric delta function dependence in time:

\begin{align*}
u_z (r,z,h, \omega )   &= ( F_1 \cos \phi\ +  F_2 \sin \phi ) ZHF + F_3 ZVF \\
 &+ \ M_{11}  \left[   \frac{ ZSS   }{ 2 }  \cos ( 2 \phi )  - \frac{ ZDD   }{ 6 } +  \frac{ ZEX   }{ 3 }\ \right] \\
 &+ \  M_{22}  \left[ \frac{ -ZSS   }{ 2 }  \cos ( 2 \phi )  - \frac{ ZDD   }{ 6 } +  \frac{ ZEX   }{ 3 }\right] \\
 &+ \  M_{33}  \left[ \frac{ ZDD   }{ 3 } +  \frac{ ZEX   }{ 3 }\right] \\
 &+ \  M_{12}  \left[ ZSS   \sin ( 2 \phi ) \right] \\
%\eqno (\*[chap].\*[sec].8a)
 &+ \  M_{13}  \left[ ZDS   \cos ( \phi ) \right] \\
 &+ \  M_{23}  \left[ ZDS   \sin ( \phi ) \right] 
\end{align*}

\begin{align*}
u_r (r,z,h, \omega )\   &=
( F_1 \cos \phi\ \,+\,\  F_2 \sin \phi ) RHF \,+\, F_3 RVF \\
 \ \ \ \ &+ \  M_{11}
\, \left[ \frac{ RSS   }{ 2 }\  \cos ( 2 \phi ) \ -\ 
\frac{ RDD   }{ 6 }\ +\  \frac{ REX   }{ 3 }\right] \\
 \ \ \ \ &+ \  M_{22} \, \left[ \frac{ -RSS   }{ 2 }\  \cos ( 2 \phi ) \ -\ 
\frac{ RDD   }{ 6 }\ +\  \frac{ REX   }{ 3 }\right] \\
 \ \ \ \ &+ \  M_{33} \, \left[ \frac{ RDD   }{ 3 }\ +\  \frac{ REX   }{ 3 }\  \right] \\
 \ \ \ \ &+ \  M_{12} \, \left[ \  RSS \  \sin ( 2 \phi ) \  \right] \\
%\eqno (\*[chap].\*[sec].8b)
 \ \ \ \ &+ \  M_{13} \, \left[ \  RDS \  \cos ( \phi ) \  \right] \\
 \ \ \ \ &+ \  M_{23} \, \left[ \  RDS \  \sin ( \phi ) \right]
\end{align*}

\begin{align*}
u_{\phi} (r,z,h, \omega )\   &= ( + F_1 \sin \phi - F_2 \cos \phi ) THF \\
 \ \ \ \ &+ \ M_{11} \, \left[ \frac{ TSS   }{ 2 }\  \sin ( 2 \phi ) \right] \\
 \ \ \ \ &+ \  M_{22} \, \left[ \frac{ -TSS   }{ 2 }\  \sin ( 2 \phi ) \right] \\
 \ \ \ \ &+ \  M_{12} \, \left[ -TSS \  \cos ( 2 \phi ) \right] \\
%\eqno (\*[chap].\*[sec].8c)
 \ \ \ \ &+ \  M_{13} \, \left[ TDS \  \sin ( \phi ) \right] \\
 \ \ \ \ &+ \  M_{23} \, \left[ -TDS \  \cos ( \phi ) \right]\ . 
\end{align*}
where we define 

\begin{align*}
ZDD \   &= \  { \frac{-1 }{ { 4 \pi \rho ( i \omega )^2 } }} \, \left[ 3 { \frac{ \partial^3 F_{\alpha}   }{  \partial {\it z}^3 } } \,+\, {{ k_{\alpha} }}^2 { \frac{ \partial F_{\alpha}   }{  \partial {\it z} } } \,-\, 3 { \frac{ \partial^3 F_{\beta}   }{  \partial {\it z}^3 } } \,-\, 3 {{ k_{\beta} }}^2 { \frac{ \partial F_{\beta}   }{  \partial {\it z} } } \right] \, \\
%\eqno (\*[chap].\*[sec].9a)
RDD \   &= \  { \frac{-1 }{ { 4 \pi \rho ( i \omega )^2 } }} \, \left[ 3 { \frac{ \partial^3 F_{\alpha}   }{  \partial {\it z}^2 \partial r } } \,-\, 3 { \frac{ \partial^3 F_{\beta}   }{  \partial {\it z}^2 \partial r } } \,+\, {{ k_{\alpha} }}^2 { \frac{ \partial F_{\alpha}   }{  \partial r } } \right] \, \\
%\eqno (\*[chap].\*[sec].9b)
ZDS \   &= \  { \frac{-1 }{ { 4 \pi \rho ( i \omega )^2 } }}  \, \left[ 2 { \frac{ \partial^3 F_{\alpha}   }{  \partial {\it z}^2 \partial r } } \,-\,2 { \frac{ \partial^3 F_{\beta}   }{  \partial {\it z}^2 \partial r } } \,-\, {{ k_{\beta} }}^2 { \frac{ \partial F_{\beta}   }{  \partial r } } \right] \, \\
%\eqno (\*[chap].\*[sec].9c)
RDS \   &= \  { \frac{-1 }{ { 4 \pi \rho ( i \omega )^2 } }} \, \left[ 2 { \frac{ \partial^3 F_{\alpha}   }{  \partial r^2 \partial {\it z} } } \,-\,2 { \frac{ \partial^3 F_{\beta}   }{  \partial r^2 \partial {\it z} } } \,-\, {{ k_{\beta} }}^2 { \frac{ \partial F_{\beta}   }{  \partial {\it z} } } \right] \, \\
%\eqno (\*[chap].\*[sec].9d)
TDS \   &= \   { \frac{1 }{ { 4 \pi \rho ( i \omega )^2 } }} \, \left[ \frac{2 }{ r }\left( { \frac{ \partial^2 F_{\alpha}   }{  \partial r \partial {\it z} } } \,-\, { \frac{ \partial^2 F_{\beta}   }{  \partial r \partial {\it z} } } \right) \,-\, {{ k_{\beta} }}^2 { \frac{ \partial F_{\beta}   }{  \partial {\it z} } } \right] \, \\
%\eqno (\*[chap].\*[sec].9e)
ZSS \   &= \   { \frac{-1 }{ { 4 \pi \rho ( i \omega )^2 } }} \, \left[ 2 { \frac{ \partial^3 F_{\alpha}   }{  \partial r^2 \partial {\it z} } } \,+\, { \frac{ \partial^3 F_{\alpha}   }{  \partial {\it z}^3 } } \,+\, {{ k_{\alpha} }}^2 { \frac{ \partial F_{\alpha}   }{  \partial {\it z} } }  \right . \\
& \left.  \ \ \ \,-\, 2 { \frac{ \partial^3 F_{\beta}   }{  \partial r^2 \partial {\it z} } } \,-\, { \frac{ \partial^3 F_{\beta}   }{  \partial {\it z}^3 } } \,-\, {{ k_{\beta} }}^2 { \frac{ \partial F_{\beta}   }{  \partial {\it z} } } \right] \, \\
%\eqno (\*[chap].\*[sec].9f)
RSS \   &= \  { \frac{-1 }{ { 4 \pi \rho ( i \omega )^2 } }} \, \left[ 2 { \frac{ \partial^3 F_{\alpha}   }{  \partial r^3 } } \,+\, { \frac{ \partial^3 F_{\alpha}   }{  \partial {\it z}^2 \partial r } } \,+\, {{ k_{\alpha} }}^2 { \frac{ \partial F_{\alpha}   }{  \partial r } } \right . \\
& \left.  \ \ \ \,-\, 2 { \frac{ \partial^3 F_{\beta}   }{  \partial r^3 } } \,-\, { \frac{ \partial^3 F_{\beta}   }{  \partial {\it z}^2 \partial r } } \,-\,2 {{ k_{\beta} }}^2 { \frac{ \partial F_{\beta}   }{  \partial r } } \right] \, \\
%\eqno (\*[chap].\*[sec].9g)
TSS \   &= \  { \frac{-1 }{ { 4 \pi \rho ( i \omega )^2 } }} \, \left[ 2 { \frac{ \partial^3 F_{\alpha}   }{  \partial r^3 } } \,+\,2 { \frac{ \partial^3 F_{\alpha}   }{  \partial {\it z}^2 \partial r } } \,+\,2 {{ k_{\alpha} }}^2 { \frac{ \partial F_{\alpha}   }{  \partial r } } \right . \\
& \left.  \ \ \ \,-\, 2 { \frac{ \partial^3 F_{\beta}   }{  \partial r^3 } }  \,-\,2 { \frac{ \partial^3 F_{\beta}   }{  \partial {\it z}^2 \partial r } } \,-\, {{ k_{\beta} }}^2 { \frac{ \partial F_{\beta}   }{  \partial r } } \right] \, \\
%\eqno (\*[chap].\*[sec].9h)
ZEX \   &= \   { \frac{1 }{  4 \pi \rho ( i \omega )^2 } } {{ k_{\alpha} }}^2 { \frac{ \partial F_{\alpha}   }{  \partial {\it z} } } \\
%\eqno (\*[chap].\*[sec].9i)
REX \   &= \   { \frac{1 }{  4 \pi \rho ( i \omega )^2 }} {{ k_{\alpha} }}^2 { \frac{ \partial F_{\alpha}   }{  \partial r } } \\
%\eqno (\*[chap].\*[sec].9j)
ZVF \   &= \  { \frac{1 }{  { 4 \pi \rho ( i \omega )^2 } } } \, \left[ { \frac{ \partial^2 F_{\alpha}   }{  \partial {\it z}^2 } } \,-\, { \frac{ \partial^2 F_{\beta}   }{  \partial {\it z}^2 } } \,-\, {{ k_{\beta} }}^2 { { F_{\beta} }} \right] \, \\
%\eqno (\*[chap].\*[sec].9k)
RVF \   &= \  { \frac{1 }{  { 4 \pi \rho ( i \omega )^2 } } } \, \left[ { \frac{ \partial^2 F_{\alpha}   }{  \partial r \partial {\it z} } } \,-\, { \frac{ \partial^2 F_{\beta}   }{  \partial r \partial {\it z} } }  \right] \, \\
%\eqno (\*[chap].\*[sec].9l)
ZHF \   &= \  { \frac{1 }{  { 4 \pi \rho ( i \omega )^2 } } } \, \left[ { \frac{ \partial^2 F_{\alpha}   }{  \partial r \partial {\it z} } } \,-\, { \frac{ \partial^2 F_{\beta}   }{  \partial r \partial {\it z} } }  \right] \, \\
%\eqno (\*[chap].\*[sec].9m)
RHF \   &= \  { \frac{1 }{  { 4 \pi \rho ( i \omega )^2 } } } \, \left[ { \frac{ \partial^2 F_{\alpha}   }{  \partial r^2 } } \,-\, { \frac{ \partial^2 F_{\beta}   }{  \partial r^2 } }  \,-\, {{ k_{\beta} }}^2 { { F_{\beta} }} \right] \, \\
%\eqno (\*[chap].\*[sec].9n)
THF \   &= \ -\, { \frac{1 }{  { 4 \pi \rho ( i \omega )^2 } } } \, \left[ { \frac{1 }{ r }} { \, \left( { \frac{ \partial F_{\alpha}   }{  \partial r } } \,-\, { \frac{ \partial F_{\beta}   }{  \partial r } } \right) \, }  \,-\, {{ k_{\beta} }}^2 { { F_{\beta} }} \right]\, \\
%\eqno (\*[chap].\*[sec].9o)
PEX \   &= \  - { \frac{{{ k_{\alpha} }}^2   }{  4 \pi   } } { { F_{\alpha} }}
%\eqno (\*[chap].\*[sec].9p)
\end{align*}

The PEX solution is the pressure field in a fluid due to a
center of expansion
source somewhere in the model.

The partial derivatives of the Sommerfeld integral are given in the Appendix.


\section{Programs}

Computer Programs in Seismology computes the $ZDD$, $RDD$, $ZDS$, $RDS$, $TDS$, $ZSS$, $RSS$, $TSS$, $ZEX$, $REX$, $ZVF$, $RVF$, $ZHF$, $RHF$, $THF$ and $PEX$ Green's functions.
Different techniques are used to make synthetics, but the meaning of the Green's functions is the same. We will focus on the wavenumber integration code, which consists of the sequence
{\bf hprep96}, {\bf hspec96} and {\bf hpulse96}. These programs prepare the data set for the wavenumber integration, evaluate the multilayered medium response in term of wavenumber and frequency and then integrate over wavenumbaer an finally, convolve a source time function, respectively. To test the {\bf hspec96} output the program {\bf whole96} implements the wholespace Green's functions given above.

The only deviation from the equations given above is that the polarity of the $u_z$ is reversed to agree with the observational convention that positive-$z$ is up on the Earth's surface. Positive-$r$ is in a radial direction away from the source, and positive-${\phi}$ is is a direction to the right when looking in the positive-$r$ direction.

For the $ZEX$ Green function, the lines in {\bf hwhole96} that define the result are:
\begin{verbatim}
c-----
c       ZEX
c-----
                 cresp(9) = cresp(9) +
     1                  (- fz(1))/
     2                  (4.0*3.1415927*rho*a*a*atna*atna)
                           ...
                 cresp(9)  = - cresp(9)
\end{verbatim}
where
\begin{verbatim}
                    ...
        xka=omega/(dble(a)*atna)
                    ...
        eye = dcmplx(0.0d+00,1.0d+00)
                    ...
        r1=dsqrt(dble(r)**2 + dble(z)**2)
                    ...
        pfac(1) = -eye * xka * dcmplx(r1,0.0d+00)
                    ...
c-----
c       Fz
c-----
        fac(1) = zr3
        fac(2) = z/r2
                    ...
        fz = -pfac* (fac(1) + kv * fac(2))
\end{verbatim}
If we put all of these pieces together, we get the following for the $ZEX$ output of {\bf hwhole96}, and by design {\bf hspec96},
\begin{equation*}
u_z(r,z,\omega)  = \frac{1}{4 \pi\rho {\alpha}^2 } \frac{1}{R} \frac{z}{R} \left[ {\frac{1}{R} + \frac{ i \omega}{\alpha} } \right] e^{ - i \omega R / \alpha }
\end{equation*} 
while the expansion of the $ZEX$ given above, after using the partial derivatives given in Appendix A, is
\begin{equation*}
u_z(r,z,\omega)  = \frac{1}{4 \pi\rho { ( i \omega} )^2 } \frac { {\omega}^2}{{\alpha}^2} \left [ \frac{z}{R^3} + \frac { i \omega}{\alpha} {\frac{z}{R^2} \right ] e^{ - i \omega R / \alpha }
\end{equation*} 
The only difference  is the difference in the definition of the positive $z$-direction which causes the change in sign.

\section{Source time function}
The derivation until now  provides the Fourier transform of the displacement due to a point force or moment tensor source in space with a Dirac delta function time dependence.
In the actual implementation of the code, a complex frequency is used, e.g., $\omega -i a $ rather than $\omega$ for several reasons. Instead of working with the Fourier transform pair
$ h(t) \Leftrightarrow H( \omega) $, we work with the transform pair
$ exp ( - a t ) h(t) \Leftrightarrow H( \omega - ia) $ which has the advantages of removing complex singularities from the real wavenumber axis but also of controlling the Discrete Fourier transform wrap-around (periodicity) that plagues numerical synthetics.  The desired $h(t)$ is obtained by multiplying the inverse transform of $ H( \omega -ia) $ by $ exp ( + a t )$.  The use of this trick also permits the use of a step function.

The program {\bf hpulse96} uses the output of {\bf hwhole96},  {\bf hspec96} or {\bf tspec96}, $ H(r,z, \omega -ia )$,  applies the source time function and yields desired time series (displacement, velocity or acceleration) to give the desired time series $h(r,z,t)$.

The simplify the discussion let $a = 0 $, let the Green function be $G(s)$, let the Fourier transform of the source time funciton be $S(s)$, and let $H(s)$ be a filter that integrates of differentiates the output, and
$s$, the Laplace transform variable, which is $s = i \omega$. Table \ref{T_1} gives some possible output for various source tiem functions and output filters.

\begin{table}
\label{T_1}
\caption{Operations}
\begin{tabular} { l l l l }
\hline
$S(s)$ & ${H(s)$ & Motion & Description \\
\hline
\\
1 & 1         & G(s) $ & displacement from impulse \\
\\
$\frac{1}{s}$ & 1 & $\frac{1}{s} G(s)$  &  displacement from step function \\
\\
$\frac{1}{s}$ & s &  G(s)$ $  & velocity from step function \\
\\
\hline
\end{tabular}
\end{table}

Table \ref{T_2} shows the result of using different command flags when running {\bf hpulse96}.

\begin{table}
\label{T_2}
\caption{Result of command line flags for {\bf hpusle96}}
\begin{tabular} { l l  }
\hline
\\
-D &  Displacement for a step source time function\\ 
\\
-V &  Velocity for a step source time function \\
\\
-A &  Acceleartion for a step source time function\\
\\
-V &  Displacement for an impulse source time function \\
\\
-V -OD  & Displacement for an impulse source time function \\
 & but the ASCII output in the file96 format indicates displacement units \\
\\
\hline
\end{tabular}
\end{table}

\section{Modern usage}
For my source studies, I use the Green's function in Sac files rather than the the ASCII {\it file96} format. I use the following syntax
to do this
\begin{verbatim}
 hprep96
 hspec96
 hpulse96 -p -V -l 1 | f96tosac -G
\end{verbatim}
This creates files with names 012500100.RDD  012500100.RDS  012500100.REX  012500100.RSS  012500100.TDS  012500100.TSS  012500100.ZDD  012500100.ZDS	012500100.ZEX  012500100.ZSS 012500100.ZVF 012500100.RVF 012500100.ZHF 012500100.RHF 012500100.THF , which represent
a ground velocity Green's function for a step-like source (I use a smooth source to avoid noise problems due to sharp truncation at the Nyquist frequency), at a distance of 125.0 km and a source depth of 10.0 km.
If I wish to make a synthetic for a specific source, I do the following in  {\bf gsac}:
\begin{verbatim}
GSAC> mt to ZRT FN 1.0e+15 FE 2.0e+15 -FZ 0.0 -AZ 122. FILE 012500100
GSAC> w
\end{verbatim}
This will create the Sac files T.Z, T.R and T.T.


To derive the static deformation for a step-like source time function, we use the relation
\begin{equation*}
h ( \infty ) = \lim_{ s \to 0 } s H(s) 
\end{equation*}
Using our Green's functions, the displacement due to a step source time function response is $H(s) = ZEX(s)/s $, for example, and the
static displacement in the limit is just $ZEX (0)$. Taking this limit very carefully, we have for the wholespace the following static deformations for a step source.

\begin{equation*}
\begin{aligned}
ZDD \,&=   \ \   \frac{1}{  4 \pi \rho } \, \left[ { \frac{2}{ \alpha^2 }\frac{z}{  R^3 } - \frac{3}{ 2 }\left( \frac{1}{ \alpha^2 }- \frac{1}{ \beta^2 }\right) \, \left( \frac{ 3 z^3  }{  R^5 } - \frac{ z  }{  R^3 } \right) } \right] \\
RDD \,&=   \ \   \frac{1}{  4 \pi \rho } \, \left[ { \frac{2}{ \alpha^2 }\frac{r}{  R^3 } - \frac{3}{ 2 }\left( \frac{1}{ \alpha^2 }- \frac{1}{ \beta^2 }\right) \, \left( \frac{ 3 r z^2  }{  R^5 } \right) - \frac{3}{ 2 }\left( \frac{1}{ \alpha^2 }+ \frac{1}{ \beta^2 }\right) \frac{r}{ R^3 }} \right] \\
ZDS \,&=   \ \   \frac{1}{  4 \pi \rho } 
 \, \left[ { \frac{1}{ \beta^2 }\frac{r}{  R^3 } + \left( \frac{1}{ \alpha^2 }- \frac{1}{ \beta^2 }\right) \,  \frac{ 3 r z^2  }{  R^5 }  -  \left( \frac{1}{ \alpha^2 }+ \frac{1}{ \beta^2 }\right) \frac{r}{ R^3 }} \right] \\
RDS \,&=  \ \   \frac{1}{  4 \pi \rho }
\left[ {
  \left( \frac{1}{ \alpha^2 }- \frac{1}{ \beta^2 }\right) \, \left( \frac{ 3 z^3  }{  R^5 } - \frac{ 2 z  }{  R^3 } \right) + \frac{ z  }{  R^3 } 
} \right] \\
TDS \,&= \ \   - \frac{1}{  4 \pi \rho } \frac{1}{ \alpha^2 }\frac{ z  }{  R^3 } \\
ZSS \,&=  \ \   - \frac{1}{  4 \pi \rho } \frac{1}{ 2 }\left( \frac{1}{ \alpha^2 }- \frac{1}{ \beta^2 }\right) \, \frac{ 3 z r^2  }{  R^5 } \\
RSS \,&=  \ \   \frac{1}{  4 \pi \rho } \frac{ R^3 {}1}{ 2 }\left[ {
\left( \frac{ 3 z^2  }{  R^2 } - 2 \right) \, \left( \frac{1}{ \alpha^2 }- \frac{1}{ \beta^2 }\right) + \left( \frac{1}{ \alpha^2 }+ \frac{1}{ \beta^2 }\right)
} \right] \\
TSS \,&=  \ \    - \frac{1}{  4 \pi \rho } \frac{1}{ \alpha^2 }\frac{r}{ R^3 } \\
ZEX \,&=  \ \   \frac{1}{  4 \pi \rho } \frac{1}{ \alpha^2 }\frac{z}{ R^3 } \\
REX \,&=  \ \   \frac{1}{  4 \pi \rho } \frac{1}{ \alpha^2 }\frac{r}{ R^3 } \\
ZVF \,&=  \ \   \frac{1}{  4 \pi \rho } \frac{1}{ 2 }\left( \frac{1}{ \alpha^2 }+ \frac{1}{ \beta^2 }\right) \frac{1}{ R }- \frac{1}{  4 \pi \rho } \frac{1}{ 2 }\left(  \frac{1}{ \alpha^2 }- \frac{1}{ \beta^2 }\right) \frac{z^2}{ R^3 } \\
RVF \,&=  \ \   - \frac{1}{  4 \pi \rho } \frac{1}{ 2 }\left( \frac{1}{ \alpha^2 }- \frac{1}{ \beta^2 }\right) \frac{rz}{ R^3 } \\
ZHF \,&=  \ \   - \frac{1}{  4 \pi \rho } \frac{1}{ 2 }\left( \frac{1}{ \alpha^2 }- \frac{1}{ \beta^2 }\right) \frac{rz}{ R^3 } \\
RHF \,&=  \ \   \frac{1}{  4 \pi \rho } \frac{1}{ 2 }\left[ { 
 - \left( \frac{1}{ \alpha^2 }- \frac{1}{ \beta^2 }\right) \frac{ r^2  }{  R^3 } 
+ \left( \frac{1}{ \alpha^2 }+ \frac{1}{ \beta^2 }\right) \frac{1}{ R }} \right] \\
THF \,&=  \ \   - \frac{1}{  4 \pi \rho } \frac{1}{ 2 }\left(  \frac{1}{ \alpha^2 }+ \frac{1}{ \beta^2 }\right) \, \frac{1}{ R }
\end{aligned}
\end{equation*}





\appendix
\begin{center}
      {\bf Appendix}
    \end{center}
\chapter{Integrals}

The Sommerfeld integral is fundamental to the representation of wave propagation in a cylindrival coordinate system. We have seen  that solutions for various combinations of points forces in a homogeneous, isotropic media, requires partial derivatives of the Sommerfeld integral with respect to the $r$ and $ z $ coordinates. This chapter provides the algebraic and integral expressions for these partial derivatives. 


\section{Partials of spherical wave propagation solution}

Given the expression for the outwardly propagating spherical wave,
$$ 
F_v \   = \  { \frac{1 }{ R }} e^{ \frac{ \,-\,i \omega R   }{ v }} \ =\  \INT
 { \frac{k }{  \nu_v } } e^{ \,-\, \nu_v |  z | } J_o ( kr ) \, dk
$$
where
$R^2 \   = \  r^2 \ \,+\,\   {  z }^2 $,
$ \nu_v^2 \,=\, k^2 - ( \omega \,/\, v )^2 $,
and
the function $  sgn(z) $ is defined as
$$
 sgn(z) = 
\begin{cases}
 0 & z < 0 \\
 1 & z \ge 0
\end{cases}
$$
then the expressions for the desired spatial derivatives of the Sommerfeld integral are as follow:
\begin{equation*}
\begin{aligned}
 \frac{ \partial F_v   }{  \partial  z }  \   &= \   \,-\,\ { e^{ \frac{ \,-\,i \omega R   }{ v }} } \, \left[  {  \frac{z }{  R^3 } }  \,+\, { \left( {\frac{ i \omega   }{ v }}  \right) }  {  \frac{z }{  R^2 } }  \right] \\
 \,&=\, \  - sgn (z) \INT  k e^{ \,-\, \nu_v |  z | } J_0 ( kr ) \, dk \\
{ \frac{ \partial F_v   }{  \partial r } } \   &= \  \,-\,\ { e^{ \frac{ \,-\,i \omega R   }{ v }} } \, \left[  { \frac{r }{  R^3 } }  \,+\, { \left( {\frac{ i \omega   }{ v }}  \right) }  { \frac{r }{  R^2 } }  \right] \\
 \,&=\, \  - \INT  \frac{ k^2   }{  \nu_v } e^{ \,-\, \nu_v |  z | } J_1 ( kr ) \, dk \\
{ \frac{ \partial^2 F_v   }{  \partial r \partial {\it z} } } \   &= \  \ { e^{ \frac{ \,-\,i \omega R   }{ v }} } \, \left[  \left( \frac{ \,3r  z   }{  R^5 } \right)  \,+\, { \left( {\frac{ i \omega   }{ v }}  \right) } \,  \left( \frac{ \,3r  z   }{  R^4 } \right)  \,+\, { \left( {\frac{ i \omega   }{ v }}  \right) }^2 \,  \left( \frac{ \,r  z   }{  R^3 } \right)  \right] \, \\
 \,&=\, \    sgn (z) \INT  { k^2 }  e^{ \,-\, \nu_v |  z | } J_1 ( kr ) \, dk \\
{ \frac{ \partial^2 F_v   }{  \partial {\it z}^2 } } \   &= \  \,+\,\ { e^{ \frac{ \,-\,i \omega R   }{ v }} } \, \left[  \left( \frac{ 3   z^2   }{  R^5 } \,-\, \frac{ 1   }{  R^3 } \right)  \,+\, { \left( {\frac{ i \omega   }{ v }}  \right) } \,  \left( \frac{ 3   z^2   }{  R^4 } \,-\, \frac{ 1   }{  R^2 } \right)  \,+\, { \left( {\frac{ i \omega   }{ v }}  \right) }^2 \,  \left( \frac{   z^2   }{  R^3 } \right)  \right] \, \\
 \,&=\, \    \INT  k \nu_v  e^{ \,-\, \nu_v |  z | } J_0 ( kr ) \, dk \\
{ \frac{ \partial^2 F_v   }{  \partial r^2 } } \   &= \  \,+\,\ { e^{ \frac{ \,-\,i \omega R   }{ v }} } \, \left[  \left( \frac{ 3 r^2   }{  R^5 } \,-\, \frac{ 1   }{  R^3 } \right)  \,+\, { \left( {\frac{ i \omega   }{ v }}  \right) } \,  \left( \frac{ 3 r^2   }{  R^4 } \,-\, \frac{ 1   }{  R^2 } \right)  \,+\, { \left( {\frac{ i \omega   }{ v }}  \right) }^2 \,  \left( \frac{ r^2   }{  R^3 } \right)  \right] \, \\
 \,&=\, \  - \INT  \frac{ k^2   }{  \nu_v } e^{ \,-\, \nu_v |  z | } \left( k J_0 ( kr ) - \frac{1 }{ r }J_1 ( kr )  \right) \, dk \\
{ \frac{ \partial^3 F_v   }{  \partial r^3 } } \   &= \ \,-\,  { e^{ \frac{ \,-\,i \omega R   }{ v }} } \, \left[  \left( \frac{\,-\,9r }{  R^5 } \,+\, \frac{ 15 r^3   }{  R^7 } \right)  \,+\, { \left( {\frac{ i \omega   }{ v }}  \right) } \,  \left( \frac{\,-\,9r }{  R^4 } \,+\, \frac{ 15 r^3   }{  R^6 } \right)  \,+\,  \right . \\
   &\left. { \left( {\frac{ i \omega   }{ v }}  \right) }^2 \,  \left( \frac{\,-\,3r }{  R^3 } \,+\, \frac{ 6 r^3   }{  R^5 } \right)  \,+\, { \left( {\frac{ i \omega   }{ v }}  \right) }^3 \,  \left( \frac{ r^3   }{  R^4 } \right)  \right] \, \\
 \,&=\, \   \INT  \frac{ k^3   }{  \nu_v } e^{ \,-\, \nu_v |  z | } \left( k J_1 ( kr ) - \frac{1 }{ r }J_2 ( kr )  \right) \, dk \\
{ \frac{ \partial^3 F_v   }{  \partial {\it z}^3 } } \   &= \  \,-\,\ { e^{ \frac{ \,-\,i \omega R   }{ v }} } \, \left[  \left( \, -\, \frac{ 9 z   }{  R^5 } \,+\, \frac{ 15   z^3   }{  R^7 } \right)  \,+\, { \left( {\frac{ i \omega   }{ v }}  \right) } \,  \left( \, -\, \frac{ 9 z   }{  R^4 } \,+\, \frac{ 15   z^3   }{  R^6 } \right)  \,+\,  \right . \\
 &\left. { \left( {\frac{ i \omega   }{ v }}  \right) }^2 \,  \left( \, -\, \frac{ 3 z   }{  R^3 } \,+\, \frac{ 6   z^3   }{  R^6 } \right)  \,+\, { \left( {\frac{ i \omega   }{ v }}  \right) }^3 \,  \left( \frac{   z^3   }{  R^4 } \right)  \right] \, \\
 \,&=\, \  - sgn (z) \INT  k \nu_v^2  e^{ \,-\, \nu_v |  z | } J_0 ( kr ) \, dk \\
{ \frac{ \partial^3 F_v   }{  \partial r^2 \partial {\it z} } } \   &= \  \,-\,\ { e^{ \frac{ \,-\,i \omega R   }{ v }} } \, \left[  \left( \,-\, \frac{ 3 z   }{  R^5 } \,+\, \frac{15 r^2   z   }{  R^7 } \right)  \,+\, { \left( {\frac{ i \omega   }{ v }}  \right) } \,  \left( \,-\, \frac{ 3 z   }{  R^4 } \,+\, \frac{15 r^2   z   }{  R^6 } \right)  \,+\, \right . \\
 &\left. { \left( {\frac{ i \omega   }{ v }}  \right) }^2 \,  \left( \,-\,  \frac{z }{  R^3 } \,+\, \frac{6 r^2   z   }{  R^5 } \right)  \,+\, { \left( {\frac{ i \omega   }{ v }}  \right) }^3 \,  \left( \frac{ r^2   z   }{  R^4 } \right)  \right] \, \\
 \,&=\, \  sgn (z) \INT  { k^2 }  e^{ \,-\, \nu_v |  z | } \left(  k J_0 ( kr ) - \frac{1 }{ r }J_1 ( kr )  \right) \, dk \\
{ \frac{ \partial^3 F_v   }{  \partial {\it z}^2 \partial r } } \   &= \  \,-\,\ { e^{ \frac{ \,-\,i \omega R   }{ v }} } \, \left[  \left( \,-\, \frac{ 3 r   }{  R^5 } \,+\, \frac{ 15 r   z^2   }{  R^7 } \right)  \,+\, { \left( {\frac{ i \omega   }{ v }}  \right) } \,  \left( \,-\, \frac{ 3 r   }{  R^4 } \,+\, \frac{ 15 r   z^2   }{  R^6 } \right)  \,+\,  \right . \\
 &\left. { \left( {\frac{ i \omega   }{ v }}  \right) }^2 \,  \left( \,-\, \frac{ r   }{  R^3 } \,+\, \frac{ 6 r   z^2   }{  R^5 } \right)  \,+\, { \left( {\frac{ i \omega   }{ v }}  \right) }^3 \,  \left( \frac{ r   z^2   }{  R^4 } \right)  \right] \, \\
 \,&=\, \   -  \INT  k^2 \nu_v  e^{ \,-\, \nu_v |  z | } J_1 ( kr ) \, dk 
\end{aligned}
\end{equation*}

\end{document}
